%%%%%%%%%%%%%%%%%%%%%%%%%%%%%%%%%%%%%%%%%%%%%%%%%%%%%%%%%%%%%%%%%%%%%%%%%%%%%%%%
% FILE: exercss-beispiel.tex
%
% This file demonstrates the use of the 'exercss' document class to provide 
% solutions for homework.
%%%%%%%%%%%%%%%%%%%%%%%%%%%%%%%%%%%%%%%%%%%%%%%%%%%%%%%%%%%%%%%%%%%%%%%%%%%%%%%%

\documentclass[
	%linenumbers,% enable line numbers
	%hyperrefcolors=false, % disable link colors
]{exercss}

% German, automatically use German words like 'Aufgabe', ...
\usepackage[ngerman]{babel}

% Provide some meta-information for this homework.
% If you specify properties as empty, like `group={}', they are hidden.
\exerciseMeta{
	course name=Lineare Algebra~I, % title
	course name short=LinA~I, % headers (from page 2 on)
	sheet number=3, % title
	group=XY1, % title
	%group={}, % uncomment this to hide the `group' property
	author=Anna Apfel, % title
	author=Bert Birne, % title
	authors short={Apfel, Birne}, % headers (from page 2 on)
}

\begin{document}
	Lösungsvorschläge zu Aufgaben können -- und sollen -- innerhalb der Umgebungen \texttt{exercise} formuliert werden. Natürlich ist es dennoch möglich, Inhalte außerhalb dieser Umgebungen zu verfassen, so wie diesen Text (vor der ersten \texttt{exercise}-Umgebung).

	Für die Demonstration einiger Layout-Eigenschaften auf der nächsten Seite, geben wir hier eine Gleichung an, die wir später referenzieren.
	\begin{equation}
		\label{euler}
		e^{i\pi} + 1 = 0
	\end{equation}
	Nun demonstriereren wir die Verwendung der \texttt{exercise}-Umgebung mitsamt ihrer Unterumgebungen \texttt{subexercises} und \texttt{subsubexercises}. Die erste Aufgabe demonstriert gleich alle drei Umgebungen. Ferner wird für die Aufgabe ein Titel, \emph{Linearität}, definiert.
	\begin{exercise}[title=Linearität]
		Da die gegebene Abbildung~$f$ linear ist, besitzt sie folgende Eigenschaften.
		\begin{subexercises}
		\item Sie ist homogen. Das heißt, es gilt $f(ax) = af(x)$ für jeden Skalar~$a$ und jeden Vektor~$x$.
		\item Außerdem ist sie additiv. Das heißt, es gilt $f(x+y) = f(x) + f(y)$ für alle Vektoren~$x$ und~$y$.

			Dazu lässt sich noch mehr sagen.
			\begin{subsubexercises}
			\item Einerseits gilt \dots
			\item Andererseits gilt \dots
			\end{subsubexercises}
		\end{subexercises}
	\end{exercise}
	Die Umgebungen \texttt{subexercises} und \texttt{subsubexercises} sind spezielle Varianten von \texttt{enumerate}-Umgebungen. Entsprechend wird jede Teil- und Teil-Teil-Aufgabe mit \verb+\item+ eingeleitet.
	\begin{exercise}
		Dieser Lösungsvorschlag besitzt keinen Titel (und auch sonst keinen Inhalt).
	\end{exercise}
	\begin{exercise}[title={$\log(e^x) = x$}]
		Auch Mathe-Umgebungen können Teil des Titels sein.
	\end{exercise}
	Nach den Aufgaben nun noch eine kurze Demonstration einiger Stilaspekte:
	\begin{itemize}
		\item \textbf{Referenzen:} Meinen persönlichen Präferenzen folgend werden Referenzen nicht durch Rechtecke eingerahmt, sondern farblich hervorgehoben.
			\begin{itemize}
				\item URL: \url{dblp.org}
				\item Labelreferenz: Gleichung~\eqref{euler} ist schön.
			\end{itemize}
		\item Weitere Aspekte folgen \dots
	\end{itemize}
\end{document}
